\section{Conclusion}

\indent Ce projet a été l'oeuvre d'une coopération de toute l'équipe. En effet nous avons dû mettre en pratique une architecture modèle-vue d'un côté
et contrôleur de l'autre. De plus, du côté contrôleur nous avons un client et un serveur, ce qui nous a permis de mettre
en pratique pour la première fois sur un projet concret nos connaissances de réseaux accumulées jusque là.\\
\indent Il a fallu réussir à emboîter la partie graphe, la partie serveur, la partie qui à la connaissance de l'état tous les
poissons, la partie client et la partie vue. Ce fût une grande épreuve, mais qui finalement s'est soldée par l'aboutissement
d'un aquarium fonctionnel (objet qui est d'ailleurs présent dans la plupart des foyers français. Qui n'a jamais rêvé de pouvoir
contrôler ses poissons en regardant son aquarium ?).\\
\indent Nous avons utilisé un large panel d'outils que nous avons appris ces dernières années : le langage C, Java, des connaissances en réseaux
mais aussi en système. L'entente au niveau de l'équipe a été cordiale : en effet chacun a travaillé professionellement sur
sa partie, et ce, de façon efficace. Le fait de rester assez proche du cahier des charges, afin d'essayer de satisfaire le client, nous
a permis d'éviter beaucoup de conflits lors de la fusion du travail de chacun.

\section{Partie controleur de la vue}

\subsection{Le prompt client}

\indent Pour ce projet, le client doit pouvoir, comme l'a été mentionné précédemment saisir des commandes au niveau du prompt client.
Le client a aussi de son côté une interface graphique.
Il faut donc pouvoir mettre dans un thread à part la saisie de commande afin de ne pas bloquer le tout.

\subsection{Le parsing de commandes saisies}

\indent Le client instancie donc une classe Commande.java qui permet de vérifier que les commandes
saisies du côté client sont bien les bonnes. Cette classe fait essentiellement du parsing de commandes
en suivant le cahier des charges du projet. Il faut savoir que les commandes éronées ne sont jamais envoyées au serveur. Seules des commandes
valides sont envoyées afin de ne pas surcharger inutilement les demandes au serveur.

\subsection{Le traitement des commandes reçues}

\indent Un peu d'une même façon, le client va en même temps instancier une classe CommandReceived.java
pour pouvoir traiter les commandes reçues. Un parseur de string est utilisé afin de déclencher des actions selon les ordres. On admet
ici que les commandes reçues sont toujours correctes.\\
Cette classe possède un attribut view, construit par référence, afin de pouvoir declencher des méthodes de la vue du client, afin d' actualiser
l'aquarium selon les commandes fournies.

\subsection{Le client}

\indent La classe client est instanciée par un Main. Elle a plusieurs attributs nécessaires à la communication avec le serveur.
On a donc un attribut socket, un buffer pour lire et envoyer sur le socket, une vue pour instancier un aquarium
et un fichier de configuration pour lire les information nécessaires à l'établissement de la connexion avec le serveur (port et adresse). \\Nous avons aussi comme nous l'avons dit plus haut, un instance de Command et CommandReceived afin de parser
et effectuer la gestion des commandes.\\
De plus nous utilisons un scanner pour récupérer les commandes saisies par l'utilisateur
dans le prompt client.\\
Cette classe possède plusieurs threads. En effet, il faut pouvoir ne pas bloquer la console en écoutant sur le serveur. De plus le scan de
la console est bloquant aussi. Il faut donc l'isoler.\\
Enfin il y a deux méthodes importantes qui permettent de recevoir sur le socket, et
aussi de pouvoir envoyer au serveur les commandes voulues : ce sont $send_message$ et $rec_message$.\\
Le client va donc faire deux actions
et ce en continu : scanner la console à la recherche d'ordres, et obéir au serveur en écoutant sur le socket.

\section{Partie Affichage}
Concernant l'affichage en lui-même, il a fallu découper cet affichage en plusieurs fichiers afin d'obtenir de clair. Les fichiers sont les suivants :
\begin{itemize}
\item{\textbf{Entity.java :} Qui comporte toputes les informations nécessaires à la description d'un poisson}
\item{\textbf{Fish.java : }Qui se charge de déssiner les poissons sur l'aquarium grâce aux informations de la classe précédente}
\item{\textbf{Aquarium.java : }Qui se charge de peindre l'aquarium et d'y placer les différents poissons décrits par la classe précédente}
\item{\textbf{View.java : }Qui se charge de gérer l'affichage dans son ensemble et de le rafraîchir}
\end{itemize}
Nous allons vous présenter un peu plus en détails ces différentes classes.
\subsection{Le modèle Poisson}
Chaque poisson possède les informations suivantes :
\begin{itemize}
\item{Des coordonnées x et y, qui correspondent à la position d'un poisson en pixels sur l'aquarium}
\item{Un nom, qui correspond à celui que lui donne l'utilisateur lors de l'ajout de celui-ci}
\item{Un type, qui permet de charger l'image correspondante lors de l'affichage si elle existe sinon une image par défaut}
\item{Des entiers représentant la longueur et la largeur en pixels du carré d'affichage de ce poisson qui sont rentrés par l'utilisateur lors de son ajout}
\end{itemize}
Ainsi grâce à ces informations contenues dans la classe \textit{Entity.java}, nous sommes capables de décrire entièrement un poisson de manière statique.


\subsection{Dessin d'un poisson}
Le dessinage d'un poisson se fait depuis la classe \textit{Fish.java} grâce aux informations receuillies de la classe \textit{Entity.java}. Cette classe est censée dessiner un composant sur l'aquarium et possède donc un lien "est-un" avec la classe \textit{JComponent}. Chaque dessin de poisson se doit d'être fidèle à la représentation attendue et possède ainsi les attributs suivants :
\begin{itemize}
\item{Des coordonnées x et y afin de l'afficher correctement sur l'aquarium}
\item{Un poissonn afin de pouvoir mettre à jour sa position sur l'écran}
\item{Une image, qui correspond à l'attribut type de la classe \textit{Entity.java}}

\end{itemize}

Le poisson peut ainsi être dessiné avec les bonnes caractéristiques tout en sachant si ce dernier a été lancé ou non.

\subsection{Dessin de l'aquarium}

Chaque aquarium quant-à lui, se charge de s'afficher et donc "est-un" \textit{JPanel}. Il se doit également de "coller" tous les dessins de poissons qui se trouvent dans cet auqarium.
\newline Ainsi un aquarium possèdera les attributs suivants :
\begin{itemize}
\item{Deux entiers correspondant à la taille de l'aquarium en pixels}
\item{Une couleur}
\item{Une liste de dessins de poissons}
\end{itemize}

Cette classe se charge en autre d'ajouter, supprimer les dessins de poissons de sa liste mais aussi de mettre à jour sa liste de poissons lors d'un rafraîchissement.

\subsection{Affichage global}

Enfin l'affichage de l'aquarium est géré depuis la classe \textit{View.java} qui se charge d'instancier un aquarium et de passer les ordres d'ajout, de suppression et de lancement d'un poisson à l'aquarium.
